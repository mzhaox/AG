
\section{Homogenization and Dehomogenization}
Let $f$ be an element of a polynomial ring. Let $\hom f$ be the homogenization of $f$ and $\dhom f$ be its dehomogenization. For an ideal $I$ of a polynomial ring, let $\hom I$ refer to the homogenization of its generators, and similarly for $\dhom I$. Clearly, both of these are still ideals. Let $P := k[x_1, ..., x_n], S := k[x_0, ..., x_n]$.

\subsection{Homogenizing and Dehomogenizing Ideals}
Clearly, the dehomogenization corresponds to quotienting by the ideal $\langle x_0 - 1 \rangle$, so dehomogenization of an ideal is still an ideal. In order to homogenize an ideal, we homogenize every element. This is clearly still an ideal, and is generated by homogenous elements (take every homogenized element as generators), so the ideal is homogeneous.

\subsection{Radical and Prime Ideals}
Let $I \subset P$. We first show that $\dhom {\hom I} = I$. If $I = \langle f_1, ..., f_m \rangle$, this follows from the fact that $\dhom {\hom {f_i}} = f_i$, since \[ 
\dhom {\hom {f_i(x_1, ..., x_n)}} = \dhom {x_0^{\deg f_i} f_i(x_1/x_0, ..., x_n/x_0)} = f_i(x_1, ..., x_n).\] 

Moreover, we can also show that $x_0^s \cdot \hom {\dhom f} = f$, for $f \in S$, where $x_0^s$ is the largest power of $x_0$ dividing $f$, since \begin{align*}
\hom {\dhom {f(x_0, ..., x_n)} } & = \hom {f(1, x_1, ..., x_n)} \\
& = x_0^{\deg \dhom f} f(1, x_1/x_0, ..., x_n/x_0) \\
& = x_0^{\deg \dhom f} \sum_{i_0, \cdots, i_n} x_1^{i_1} \cdots x_n^{i_n} / x_0^{i_1 + \cdots i_n} \\
& = \sum_{i_0 + \cdots + i_n = \deg \dhom f} x_0^{i_0} \cdots x_n^{i_n} = f x_0^{-s}.
\end{align*}
Note that the key point here is that $\deg \dhom f$ may not equal $\deg f$, i.e. by evaluating $x_0 = 1$, we may have reduced the largest degree in $f$. This shows that in general, for $I \subset S$, $\hom \dhom I \neq I$. Explicitly, if $I = \langle f_1, ..., f_m\rangle$, we see that $x_0^{s_i} \cdot \hom {\dhom {f_i}} = f_i$, so $I \subset \hom {\dehom I}$.
\subsubsection{Prime Ideals} Dehomogenization is quotienting by the kernel of an evaluation map sending a variable to 1. Prime ideals are preserved by dehomogenization since prime ideals are preserved by quotienting. When homogenizing a prime ideal $I$, take $fg \in \hom{I}$. Then $\dehom f \ \dehom g \in I$, and so WLOG $\dehom f \in I$. Then $x_0^{-s} f \in \hom I$, so $f \in \hom I$.

On the other hand, if we have a radical ideal $I$, then for $f \in \sqrt{\hom I}$, there exists $n$ such that $f^n \in \hom I$. But then $\dehom{f}^n \in I$ and so $\dehom f \in I$. Then applying $\mathrm{hom}$, we find that $x_0^{-s} f \in \hom I$, so $f \in \hom I.$ A similar reasoning can be applied for showing the dehomogenization of a radical ideal is radical.

\section{Twisted Cubic}
\subsection{Inclusion of Affine Algebraic Sets into Projective Algebraic Sets} Given an algebraic set $X$ in affine space, we can consider $V_h(\hom {I(X)})$, which is a closed algebraic set in projective space. So an algebraic subset of projective space must at least be closed.

\subsection{Twisted Cubic}
We claim that $C = V(y - x^2, xy - z)$. It is clear points of the form $(t, t^2, t^3)$ kill those functions. On the other hand, if $y = x^2,$ then $z = xy = x^3$, which describes a point on $C$. Hence $I(C)$ is generated by $y - x^2, xy - z$. On the other hand, we also have $xz - y^2 = -y(y - x^2) - x(xy - z) \in I(C)$. However, if we homogenize, $xz - y^2$ is fixed and the generators of $I(C)$ are sent to $wy - x^2$ and $xy - wz$.

However, if we try to write $xz - y^2 = f(w,x,y,z) (wy - x^2) + g(w,x,y,z)(xy - zw),$ then based on the homogenous components, $f, g$ must be constants, giving us a contradiction. Hence, three generators is not enough for $I(\bar{C}).$

\section{Singular Points}
\subsection{Variety and Its Projective Closure}
Let $P, S$ be as before, and let $I \subset P$ be an ideal. Let $\hom I$ be as before. Then we send $(f + I)/(g + I) \in P/I \rightarrow (\hom f + \hom I)/(\hom g + \hom I)$. Denote this map by $\home^*$. The reverse map we define by $(a + \hom I)/(b + \hom I) \rightarrow (\dhom a + I)/(\dhom b + I).$ Clearly, we have $\dehom^* \circ \home^*$ as the identity. Not quite sure how to do the other direction.

\subsection{Singular iff Singular in Projective Closure}

\subsection{Singular Points on Curve}

\subsection{Singular Points on Cubic}
Since $C$ is a plane curve, the closure is just given by the homogenization: $\bar{C} = V(z^2 y - x^3).$ This has gradient $\langle -3x^2, z^2, 2zy \rangle$, which is singular at $(0, 1, 0)$.

\section{Quadratic Forms}
To compute the $i$-th gradient component, let $F = \sum_{i, j} c_{ij} x_i x_j$. Then $F_i = 2 c_{ii} x_i + 2 \sum_{j \neq i} c_{ij} x_j$. Notice that this entry can be written as \[(x_1 \ x_2 \cdots x_n) \cdot (2c_{i1} \ \cdots 2c_{ii} \cdots 2c_{in}),\] where $\cdot$ denotes the dot product. If we let $\bar{x} = (x_1 \ x_2 \cdots x_n)$, then the gradient can be written as $\bar{x} (c_{ij})^T = \bar{x} (c_{ij}).$ Then the singular locus of this is exactly those $\bar{x}$ such that $\bar{x} (c_{ij}) \bar{x}^T = 0,$ as desired.