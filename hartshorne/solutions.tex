\documentclass[10pt,answers,addpoints]{exam}
% Symbols
\usepackage{amsmath, amsthm, amssymb}
\usepackage{mathtools}

% Style
\usepackage[margin=1in]{geometry}
\usepackage{parskip}
\usepackage[shortlabels]{enumitem}

% Colored text
\usepackage{color}

% Links
\usepackage[bookmarks,colorlinks,breaklinks]{hyperref}
% PDF hyperlinks, with coloured links
\hypersetup{linkcolor=red,citecolor=blue,
  filecolor=dullmagenta,urlcolor=darkblue}

% Bibliography and references
\usepackage[numbers]{natbib}
\usepackage{cleveref}

% Figures
\usepackage{graphicx}


\makeatletter
\def\thm@space@setup{%
  \thm@preskip=0.5em\thm@postskip=\thm@preskip%
}
\makeatother
% Unnamed theorem without counter
\newtheorem*{theorem}{Theorem}
\newtheoremstyle{named}{}{}{\\itshape}{}{\bfseries}{.}{.5em}{\thmnote{#3's }#1}
\theoremstyle{named}
% Named theorem without counter
\newtheorem*{namedtheorem}{Theorem}
% Other theorem environments
\theoremstyle{plain}
\newtheorem{thm}{Theorem}[section]
\newtheorem{defn}[thm]{Definition}
\newtheorem{prop}[thm]{Proposition}
\newtheorem{lem}[thm]{Lemma}
\newtheorem{exmpl}[thm]{Example}
\crefformat{footnote}{#2\footnotemark[#1]#3}

\theoremstyle{remark}
\newtheorem{rmk}[thm]{Remark}

\input{commands/algebra}
\input{commands/geometry}
\input{commands/iso}
\input{commands/ktheory}
\input{commands/linalg}
\input{commands/numbers}
\input{commands/representations}



\begin{document}

\section{Varieties}
\subsection{Affine Varieties}

\begin{questions}
\question Let $k = \conj{k}$ be a field.
\begin{parts}
\part Let $Y = \{(x, x^2) | x \in k\}$. Show that $A(Y)$ is isomorphic to a
polynomial ring in one variable over $k$.
\begin{solution}
Define $\varphi : k[x,y] \rightarrow k[x]$ by $y \rightarrow x^2$. This is
clearly surjective, and we just want to show that $\ker \varphi = I(Y)$. If $f
\in \ker \varphi$, then $f(x, x^2)$ is identically zero, so $f$ vanishes at
every point of $Y$. If $f \in I(Y)$, so that $f(a, a^2)$ for every $a \in k$, we
can only conclude $f(x, x^2) = 0$ since $k$ is algebraically closed. This
follows from the
\begin{lem}
For $f_1, f_2 \in k[x]$, if $f_1(a) = f_2(a)$ for every $a \in k$, then $f_1 =
f_2$.
\end{lem}
\begin{proof}
Since $k$ is algebraically closed, $f_1 - f_2$ has finitely many zeros, unless
it is identically zero. It has infinitely many zeros since $k$ is
algebraically closed. Thus, $f_1 = f_2$.
\end{proof}
Thus, $k[x,y] / I(Y) \iso k[x]$.
\end{solution}
\part Let $Z = \{ (x, y) | xy = 1\}$. Show that $A(Z)$ is not isomorphic to a
polynomial ring in one variable over $k$.
\begin{solution}
Since $xy - 1$ is irreducible, $(xy - 1)$ is prime, and $I(Z) = (xy - 1)$. Then
$A(Z) = k[x,y]/(xy - 1)$. Suppose there is an isomorphism $\phi: A(Z)
\rightarrow k[x,y]$. Then $\phi(x) = p(t), \phi(y) = q(t)$. Since $xy = 1$, $p
(t)q(t) = 1$. Since $k$ is algebraically closed, this implies $p, q$ are
constants, and so $\phi$ is not surjective.
\end{solution}
\part Let $f$ be an irreducible quadratic polynomial in $k[x,y]$, and let $W$ be
the conic defined by $f$. Show that $A(W)$ is isomorphic to $A(Y)$ or $A(Z)$.
Which one is it and when?
\end{parts}
\question Let $Y = \{ (t, t^2, t^3) | t \in k\}$. Show that $Y$ is an affine
variety of dimension 1. Find generators for $I(Y)$. Show that $A(Y)$ is
isomorphic to a polynomial ring in one variable over $k$.
\begin{solution}
Clearly $Y = Z((y - x^2) + (z - x^3)) = Z(y - x^2) \cap Z(z - x^3)$. Then we
want to show that $(y - x^2, z - x^3)$ is a prime ideal. But the quotient of $k
[x,y,z]$ by this ideal is $k[x]$, a domain. Thus, $I(Y) = (y - x^2, z - x^3)$.
\end{solution}

\question Let $Y$ be the algebraic set defined by $x^2 - yz$ and $xz - x$. Show
that $Y$ is a union of three irreducible components. Describe them and find
their prime ideals.
\begin{solution}
Since \begin{align*}
Y & = Z(x^2 - yz) \cap Z(xz - x) \\
& = Z(x^2 - yz) \cap (Z(x) \cup Z(z - 1)) \\
& = (Z(x^2 - yz) \cap Z(x)) \cup (Z(x^2 - yz) \cap Z(z - 1)) \\
& = (Z(x) \cap Z(y)) \cup (Z(x) \cap Z(z)) \cup (Z(x^2 - y) \cap Z(z - 1)).
\end{align*}
These components have prime ideals $(x, y)$, $(x, z)$, $(x^2 - y, z - 1)$. These
components all have coordinate rings isomorphic to $k[t]$. Thus they are
indeed prime.
\end{solution}

\question If we identify $\mathbb{A}^2$ with $\mathbb{A}^1 \times \mathbb{A}^1$,
show that the Zariski topology on $\mathbb{A}^2$ is not the product topology.
\begin{solution}
All closed sets in the product topology are either (i) empty or the whole plane,
(2) collections of finite points, or (3) finite collections of horizontal and
vertical lines. However, by definition of the Zariski topology on $\mathbb
{A}^2$, $Z(y - x^2)$ is closed and is not in either of the three categories.
\end{solution}

\question Show that a $k$-algebra is isomorphic to the affine coordinate ring of
some algebraic set in some affine space if and only if $B$ is a finitely
generated $k$-algebra with no nilpotent elements.
\begin{solution}
Suppose $B$ has $n$ generators. Then there is $\varphi: k[x_1, ..., x_n]
\twoheadrightarrow B$, sending $x_i$ to the $i$-th generator. We just need to
show that $\ker \varphi$ is a radical ideal, so that it corresponds to an
algebraic set. If $f^n \in \ker \varphi$, then $\varphi(f)^n = 0$. Since $B$ has
no nilpotents, $\varphi(f) = 0$, hence $f \in \ker \varphi$.
\end{solution}

\question Any nonempty open subset of an irreducible topological space is dense
and irreducible. If $Y$ is a subset of a topological space $X$, which is
irreducible in its induced topology, then the closure $\conj{Y}$ is also
irreducible.
\begin{solution}
Let $X$ be an irreducible topological space, and $U$ be a nonempty open
subset. Then the closure $\conj{U}$ of $U$ is a closed set, and $X = \conj{U}
\cup (X \setminus U)$. Thus, $X \setminus U$ is either empty or all of $X$.
However, $U$ is non-empty, so it can't be all of $X$. Thus it is empty, and
$\conj{U} = X$, so $U$ is dense.

Write $U = (C_1 \cap U) \cup (C_2 \cap U)$ for $C_1, C_2$ closed sets of $X$.
Assume that both $C_i \cap X$ are proper subsets of $U$.
Then taking the closure of both sides, we see that $\conj{C_1 \cap U}$ is either
empty or all of $X$. If it is empty, then $C_1 \cap U$ is empty, a
contradiction. Thus it must be all of $X$, and $\conj{C_2 \cap U}$ must be
empty. However, this again produces a contradiction. Thus, $U$ must be
irreducible.

Now assume $Y$ is irreducible. If $\conj{Y}$ isn't, then
$\conj{Y} = (C_1 \cap \conj{Y}) \cup (C_2 \cap \conj{Y})$, where $C_i \cap
\conj{Y}$ are both proper subsets of $\conj{Y}$ and $C_i$ are both closed sets
of $X$. Intersecting both sides with $Y$, we obtain $Y = (C_1 \cap Y) \cup (C_2
\cap Y)$. Then one of these equals $Y$, and without loss of generality suppose
it is $C_1 \cap Y$. Then $Y \subset C_1$, so $\conj{Y} \subset C_1$, since $C_1$
is closed. This contradicts the assumption that $C_i \cap \conj{Y}$ are both
proper subsets of $\conj{Y}$.
\end{solution}


\question
\begin{parts}
\part Show that the following conditions are equivalent for a topological space
$X$:
\begin{enumerate}[(i)]
\item $X$ is noetherian,
\item every nonempty family of closed subsets has a minimal element
\item $X$ satisfies the ascending chain condition for open subsets,
\item every nonempty family of open subsets has a maximal element
\end{enumerate}

\part A noetherian topological space is quasi-compact.

\part Any subset of a noetherian topological space is noetherian in its induced
topology.

\part A noetherian space which is also  Hausdorff must be a finite set with the
discrete topology.
\end{parts}
\question Let $Y$ be an affine variety of dimension $r$. Let $H$ be a
hypersurface in the same affine space, and assume $Y \not\subseteq H$. Then
every irreducible component of $Y \cap H$ has dimension $r - 1$.
\end{questions}

\bibliographystyle{amsplain}
\bibliography{refs}
\end{document}

