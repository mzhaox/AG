\documentclass[12pt,reqno]{amsart}
% Symbols
\usepackage{amsmath, amsthm, amssymb}
\usepackage{mathtools}

% Style
\usepackage[margin=1in]{geometry}
\usepackage{parskip}
\usepackage[shortlabels]{enumitem}

% Colored text
\usepackage{color}

% Links
\usepackage[bookmarks,colorlinks,breaklinks]{hyperref}
% PDF hyperlinks, with coloured links
\hypersetup{linkcolor=red,citecolor=blue,
  filecolor=dullmagenta,urlcolor=darkblue}

% Bibliography and references
\usepackage[numbers]{natbib}
\usepackage{cleveref}

% Figures
\usepackage{graphicx}


\makeatletter
\def\thm@space@setup{%
  \thm@preskip=0.5em\thm@postskip=\thm@preskip%
}
\makeatother
% Unnamed theorem without counter
\newtheorem*{theorem}{Theorem}
\newtheoremstyle{named}{}{}{\\itshape}{}{\bfseries}{.}{.5em}{\thmnote{#3's }#1}
\theoremstyle{named}
% Named theorem without counter
\newtheorem*{namedtheorem}{Theorem}
% Other theorem environments
\theoremstyle{plain}
\newtheorem{thm}{Theorem}[section]
\newtheorem{defn}[thm]{Definition}
\newtheorem{prop}[thm]{Proposition}
\newtheorem{rmk}[thm]{Remark}
\newtheorem{lem}[thm]{Lemma}
\newtheorem{exmpl}[thm]{Example}
\crefformat{footnote}{#2\footnotemark[#1]#3}

% A large, noticeable quotient
\newcommand{\quotient}[2]{{\raisebox{.2em}{$#1$}
                           \left/\raisebox{-.2em}{$#2$}\right.}}


% Affine Grassmannian of a group
\newcommand{\gr}[1]{\mathrm{Gr}_{#1}}

% Category of perverse sheaves
\newcommand{\perv}{\mathrm{Perv}}


% Regular-sized isomorphism.
\newcommand{\iso}{\simeq}

% Experimental big isomorphism commands.

% Bigger isomorphism (with arrow).
\newcommand{\longiso}{\stackrel{\sim}{\longrightarrow}}

% Very big isomorphism (no arrow).
\makeatletter
\newcommand*{\isomorphism}{%
  \mathrel{%
    \mathpalette\@isomorphism{}%
  }%
}
\newcommand*{\@isomorphism}[2]{%
  % Calculate the amount of moving \sim up as in \simeq
  \sbox0{$#1\simeq$}%
  \sbox2{$#1\sim$}%
  \dimen@=\ht0 %
  \advance\dimen@ by -\ht2 %
  %
  % Compose the two symbols
  \sbox0{%
    \lower1.9\dimen@\hbox{%
      $\m@th#1\relbar\isomorphism@joinrel\relbar$%
    }%
  }%
  \rlap{%
    \hbox to \wd0{%
      \hfill\raise\dimen@\hbox{$\m@th#1\sim$}\hfill
    }%
  }%
  \copy0 %
}
\newcommand*{\isomorphism@joinrel}{%
  \mathrel{%
    \mkern-3.4mu %
    \mkern-1mu %
    \nonscript\mkern1mu %
  }%
}
\makeatother

% Grothendieck group
\newcommand{\ggroup}{K_0}


% GL_n
\newcommand{\gl}[2]{\mathrm{GL}_{#1}({#2})}
% SL_n
\renewcommand{\sl}[2]{\mathrm{SL}_{#1}({#2})}
% SO_n
\newcommand{\so}[2]{\mathrm{SO}_{#1}({#2})}
% Sp_n
\renewcommand{\sp}[2]{\mathrm{Sp}_{#1}({#2})}

% Trace
\newcommand{\tr}{\operatorname{tr}}


% Conjugation
\newcommand{\conj}[1]{\overline{#1}}
% Class group
\newcommand{\cl}[1]{{\mathrm{Cl}{(#1)}}}
% Order of a field
\newcommand{\roi}[1]{\mathcal{O}_{#1}}
% Galois group of L/K
\newcommand{\gal}[2]{\mathrm{Gal}{(#1 / #2)}}

% Rationals
\newcommand{\rats}{\mathbb{Q}}
% Real numbers
\newcommand{\reals}{\mathbb{R}}
% Nonnegative real numbers
\newcommand{\nnreals}{\mathbb{R}^{\geq{0}}}
% Complex numbers
\newcommand{\cmplx}{\mathbb{C}}
% Integers
\newcommand{\ints}{\mathbb{Z}}
% p-adic rationals
\newcommand{\Qp}{\rats_p}
\newcommand{\Zp}{\ints_p}

% Algebraic number theory
%  Prime ideals
\newcommand{\prim}{\mathfrak{p}} % \prime is the symbol: '
\newcommand{\oprime}{\mathfrak{P}}
\newcommand{\oprimealt}{\oprime^o}

% Frobenius map
\newcommand{\frob}[1]{\mathrm{Fr}_{#1}}
% Legendre symbol
\newcommand{\legndr}[2]{\left(\dfrac{#1}{#2}\right)}
% Norm, with the fields understood
\newcommand{\N}[1]{N{(#1)}}

% Integers modulo a given number
\newcommand{\Zmod}[1]{\mathbb{Z}/#1 \mathbb{Z}}
% Units of integers mod n
\newcommand{\Zmodu}[1]{\left(\mathbb{Z}/#1 \mathbb{Z}\right)^\times}




% Langlands dual
\newcommand{\langdual}[1]{{#1}^\vee}

% Category of representations
\newcommand{\rep}{\mathrm{Rep}}





\begin{document}

\section{Varieties}
\subsection{Affine Varieties}
Several of the first questions will need the following result, which describes
when a polynomial is identically zero.

\begin{lem}\label{ident-zero}
Let $A := k[x_1, ..., x_n]$ be a polynomial ring. If the image of $f$ is zero
under $A \rightarrow A / \mathfrak{m}$ for every maximal ideal $\mathfrak{m}$,
then $f$ is zero.
\end{lem}
\begin{proof}
This follows from the Jacobson radical of $A$ being zero (for any field, not
necessarily algebraically closed). 
\end{proof}

For $k$ algebraically closed, the maximal ideals correspond to points
in $\mathbb{A}^k$, so a polynomial which evaluates to 0 for every element of $k$
is identically zero. However, if $k$ is finite, then while the Jacobson radical
of $A$ is zero, the ideal
\[
\bigcap_{(a_1, ..., a_n) \in k^n} \langle x_1 - a_1, ..., x_n - a_n \rangle
\neq 0,
\]
in general.

\begin{questions}
\question Let $k = \conj{k}$ be a field.
\begin{parts}
\part Let $Y = \{(x, x^2) | x \in k\}$. Show that $A(Y)$ is isomorphic to a
polynomial ring in one variable over $k$.
\begin{solution}
Define $\varphi : k[x,y] \rightarrow k[x]$ by $y \rightarrow x^2$. This is
clearly surjective, and we just want to show that $\ker \varphi = I(Y)$. If $f
\in \ker \varphi$, then $f(x, x^2)$ is identically zero, so $f$ vanishes at
every point of $Y$. If $f \in I(Y)$, so that $f(a, a^2)$ for every $a \in k$,
by lemma \ref{ident-zero}, we can conclude $f(x, x^2) = 0$. Thus,
$k[x,y] / I(Y) \iso k[x]$.
\end{solution}
\part Let $Z = \{ (x, y) | xy = 1\}$. Show that $A(Z)$ is not isomorphic to a
polynomial ring in one variable over $k$.
\begin{solution}
Since $xy - 1$ is irreducible, $(xy - 1)$ is prime, and $I(Z) = (xy - 1)$. Then
$A(Z) = k[x,y]/(xy - 1)$. Suppose there is an isomorphism $\phi: A(Z)
\rightarrow k[x,y]$. Then $\phi(x) = p(t), \phi(y) = q(t)$. Since $xy = 1$, $p
(t)q(t) = 1$. Since $k$ is algebraically closed, this implies $p, q$ are
constants, and so $\phi$ is not surjective.
\end{solution}
\part Let $f$ be an irreducible quadratic polynomial in $k[x,y]$, and let $W$ be
the conic defined by $f$. Show that $A(W)$ is isomorphic to $A(Y)$ or $A(Z)$.
Which one is it and when?
\end{parts}
\question Let $Y = \{ (t, t^2, t^3) | t \in k\}$. Show that $Y$ is an affine
variety of dimension 1. Find generators for $I(Y)$. Show that $A(Y)$ is
isomorphic to a polynomial ring in one variable over $k$.
\begin{solution}
Clearly $Y = Z((y - x^2) + (z - x^3)) = Z(y - x^2) \cap Z(z - x^3)$. Then we
want to show that $(y - x^2, z - x^3)$ is a prime ideal. But the quotient of $k
[x,y,z]$ by this ideal is $k[x]$, a domain. Thus, $I(Y) = (y - x^2, z - x^3)$.
\end{solution}

\question Let $Y$ be the algebraic set defined by $x^2 - yz$ and $xz - x$. Show
that $Y$ is a union of three irreducible components. Describe them and find
their prime ideals.
\begin{solution}
Since \begin{align*}
Y & = Z(x^2 - yz) \cap Z(xz - x) \\
& = Z(x^2 - yz) \cap (Z(x) \cup Z(z - 1)) \\
& = (Z(x^2 - yz) \cap Z(x)) \cup (Z(x^2 - yz) \cap Z(z - 1)) \\
& = (Z(x) \cap Z(y)) \cup (Z(x) \cap Z(z)) \cup (Z(x^2 - y) \cap Z(z - 1)).
\end{align*}
These components have prime ideals $(x, y)$, $(x, z)$, $(x^2 - y, z - 1)$. These
components all have coordinate rings isomorphic to $k[t]$. Thus they are
indeed prime.
\end{solution}

\question If we identify $\mathbb{A}^2$ with $\mathbb{A}^1 \times \mathbb{A}^1$,
show that the Zariski topology on $\mathbb{A}^2$ is not the product topology.
\begin{solution}
All closed sets in the product topology are either (i) empty or the whole plane,
(2) collections of finite points, or (3) finite collections of horizontal and
vertical lines. However, by definition of the Zariski topology on $\mathbb
{A}^2$, $Z(y - x^2)$ is closed and is not in either of the three categories.
\end{solution}

\question Show that a $k$-algebra is isomorphic to the affine coordinate ring of
some algebraic set in some affine space if and only if $B$ is a finitely
generated $k$-algebra with no nilpotent elements.
\begin{solution}
Suppose $B$ has $n$ generators. Then there is $\varphi: k[x_1, ..., x_n]
\twoheadrightarrow B$, sending $x_i$ to the $i$-th generator. We just need to
show that $\ker \varphi$ is a radical ideal, so that it corresponds to an
algebraic set. If $f^n \in \ker \varphi$, then $\varphi(f^n) = \varphi(f)^n =
0$. Since $B$ has no nilpotents, $\varphi(f) = 0$, hence $f \in \ker \varphi$.
\end{solution}

\question Any nonempty open subset of an irreducible topological space is dense
and irreducible. If $Y$ is a subset of a topological space $X$, which is
irreducible in its induced topology, then the closure $\conj{Y}$ is also
irreducible.
\begin{solution}
Let $X$ be an irreducible topological space, and $U$ be a nonempty open
subset. Then the closure $\conj{U}$ of $U$ is a closed set, and $X = \conj{U}
\cup (X \setminus U)$. Thus, $X \setminus U$ is either empty or all of $X$.
However, $U$ is non-empty, so it can't be all of $X$. Thus it is empty, and
$\conj{U} = X$, so $U$ is dense.

Write $U = (C_1 \cap U) \cup (C_2 \cap U)$ for $C_1, C_2$ closed sets of $X$.
Assume that both $C_i \cap X$ are proper subsets of $U$.
Then taking the closure of both sides, we see that $\conj{C_1 \cap U}$ is either
empty or all of $X$. If it is empty, then $C_1 \cap U$ is empty, a
contradiction. Thus it must be all of $X$, and $\conj{C_2 \cap U}$ must be
empty. However, this again produces a contradiction. Thus, $U$ must be
irreducible.

Now assume $Y$ is irreducible. If $\conj{Y}$ isn't, then
$\conj{Y} = (C_1 \cap \conj{Y}) \cup (C_2 \cap \conj{Y})$, where $C_i \cap
\conj{Y}$ are both proper subsets of $\conj{Y}$ and $C_i$ are both closed sets
of $X$. Intersecting both sides with $Y$, we obtain $Y = (C_1 \cap Y) \cup (C_2
\cap Y)$. Then one of these equals $Y$, and without loss of generality suppose
it is $C_1 \cap Y$. Then $Y \subset C_1$, so $\conj{Y} \subset C_1$, since $C_1$
is closed. This contradicts the assumption that $C_i \cap \conj{Y}$ are both
proper subsets of $\conj{Y}$.
\end{solution}


\question
\begin{parts}
\part Show that the following conditions are equivalent for a topological space
$X$:
\begin{enumerate}[(i)]
\item $X$ is noetherian,
\item every nonempty family of closed subsets has a minimal element
\item $X$ satisfies the ascending chain condition for open subsets,
\item every nonempty family of open subsets has a maximal element
\end{enumerate}
\begin{solution}
The key point is to understand that a minimal element contains no other elements
from the family, and that a maximal element isn't contained in any other element
from the family.

Assume $X$ is noetherian, and consider a nonempty family of closed subsets.
Choose an element $C_1$ from this family. If it is minimal, we are done.
Otherwise, choose $C_2 \subset C_1$. Repeat the process for $C_2$, and so
on. Such a process will stabilize, otherwise we can obtain a descending chain
that violates the descending chain condition.

Now assume (ii), and consider an ascending chain of open sets. Taking the
complements gives us a descending chain of closed sets. This must have a minimal
element, thus the chain must stabilize. By taking complements, we see that the
ascending chain condition for open sets is satisfied for the original chain.

Now assume that $X$ satisfies the ascending chain condition on open subsets.
Then repeat the proof of (i) $\Rightarrow$ (ii).

Now assume that every nonempty family of open subsets of $X$ has a maximal
element. Then repeat the proof of (ii) $\Rightarrow$ (iii).
\end{solution}

\part A noetherian topological space is quasi-compact.
\begin{solution}
Take a cover of a noetherian space $X$ by $\{U_\alpha\}_{\alpha \in I}$.
Consider the family of finite unions of $U_\alpha$, for $\alpha \in I$. Then
there is a maximal element $U_1 \cup \cdots \cup U_n$. Suppose that this does
not equal $X$. Then there is a point lying outside of this union, and since $
\{U_\alpha\}$ are a cover, there is a $U_\beta$ containing this point. 
Then $U_1 \cup \cdots \cup U_n \cup U_\beta$ is a finite union larger
than $U_1 \cup \cdots \cup U_n$, a contradiction.
\end{solution}

\part Any subset of a noetherian topological space is noetherian in its induced
topology.
\begin{solution}
Let $Y \subset X$ be a subset of a noetherian space $X$. Suppose we have a
descending chain of closed sets $Z_1 \supset Z_2 \supset \cdots$, where the
$Z_i$ are closed in $Y$. They are written as $Z_i = C_i \cap Y$, for $C_i$
closed in $X$. Now take $D_i = C_1 \cap \cdots \cap C_i$. Then $D_i \supset D_
{i + 1}$, so the $D_{i}$ form a descending chain of closed subsets and must
stabilize, i.e. $D_r = D_{r + 1}$. Then $C_{r + 1} \supset C_1 \cap \cdots \cap
C_r$. Intersecting with $Y$, we obtain $Z_{r + 1} \supset \bigcap_{i = 1}^r
Z_i = Z_r$. Since $Z_r \supset Z_{r + 1}$, $Z_r = Z_{r + 1}$. By induction, the
$Z_i$ stabilize past $Z_r$. Thus $Y$ is Noetherian.
\end{solution}

\part A noetherian space which is also Hausdorff must be a finite set with the
discrete topology.
\begin{solution}
If $X$ is noetherian, it can be written as a union of maximal irreducible,
closed subsets $Z_i$. Each $Z_i$ must be Noetherian by the previous part, and
will of course be Hausdorff.

In this case, assume that $Z_i$ has more than one point, and take two distinct
points $x, y$. Then there are neighborhoods $U, V$ of $x, y$ such that $U \cap
V = \emptyset$. However, this contradicts the formulation of irreducibility as
``any two non-empty opens have non-empty intersection.'' Thus $Z_i$ consists of
a single point.

Thus, all points of $X$ are closed, hence any subset of $X$ is closed, being a
finite union of closed sets. Thus $X$ had the discrete topology.
\end{solution}
\end{parts}
\question Let $Y$ be an affine variety of dimension $r$ in $\mathbb{A}_k^n$. Let
$H$ be a hypersurface in the same affine space, and assume $Y \not\subseteq H$.
Then every irreducible component of $Y \cap H$ has dimension $r - 1$.
\begin{solution}
Let $Y$ correspond to the $k$-algebra domain $B = k[x_1, ..., x_n]/I(Y)$. Then
$H = Z(f)$ for some irreducible $f \in k[x_1, ..., x_n]$. Then $\dim B = r$, and
$I(H) = I(Z(f)) = (f)$. Since $Y \not\subseteq H$, $f \not \in I(Y)$, and so in
$B$, $f$ is non-zero.

Now take an irreducible component $W$ of $Y \cap H$. Then $W \subset Y, H$, so
$I(W) \supset (f), I(Y)$. Thus $I(W)$ corresponds to a prime ideal $\mathfrak
{p}$ in $B$ containing $f$. By Krull's Hauptidealsatz, $\mathfrak{p}$ has height
$1$. Then $\dim B/\mathfrak{p} = r - 1$. By the third isomorphism theorem,
$B/\mathfrak{p}$ is indeed the coordinate ring of $W$, hence $\dim W = r - 1$.
\\
\begin{rmk}
Note that $Y$, $H$ are both closed, so their intersection is closed. Then they
are a union of irreducible closed subsets (of affine space, irreducible when
those subsets are given the induced topology). Thus, $W$ is a closed subset of
the ambient affine space, and irreducible. So $I(W)$ is prime.
\end{rmk}
\end{solution}

\question Let $\mathfrak{a} \subseteq A = k[x_1, ..., x_n]$ be an ideal which
can be generated by $r$ elements. Then every irreducible component of $Z
(\mathfrak{a})$ has dimension at least $n - r$.
\begin{solution}
Let $a = \langle f_1, ..., f_r \rangle$. Then $Z(\mathfrak{a}) = Z(f_1) \cap
\cdots \cap Z(f_r)$. Let $f_i = \prod_{j = 1}^{n_i} f_{ij}$ be a factorization
into irreducibles. Then
\begin{align*}
Z(\mathfrak{a}) & = \bigcap_{i = 1}^r \bigcup_{j = 1}^{n_i} Z(f_ij) \\
& = \bigcup_{(j_1, ..., j_r)} Z(f_{ij_1}) \cap \cdots \cap Z(f_{ij_r})
\end{align*}
By the previous exercise, the irreducible components of each of the terms of the
union is at least $n - r$ (there may be some redundancies among the
irreducibles).
\end{solution}

\question
\begin{parts}
\part If $Y$ is any subset of a topological space $X$, then $\dim Y \leq \dim X$.
\begin{solution}
\end{solution}
\part If $X$ is a topological space which is covered by a family of open
subsets $\{U_i\}$, then $\dim X = \sup \dim U_i$.
\part Give an example of a topological space $X$ and a dense open subset $U$
with $\dim U < \dim X$.
\part If $Y$ is a closed subset of an irreducible finite-dimensional tpological
spzce $X$, and if $\dim Y = \dim X$, then $Y = X$.
\part Give an example of a noetherian topological space of infinite dimension.
\end{parts}

\end{questions}

\bibliographystyle{amsplain}
\bibliography{refs}
\end{document}

